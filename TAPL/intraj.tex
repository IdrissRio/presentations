\documentclass[usenames,dvipsnames]{beamer}
%-------------------------------------------------------
% THEME SETTINGS
%-------------------------------------------------------
\usetheme[progressstyle=movingCircCnt]{Feather}
\setbeamercolor{Feather}{fg=black!30,bg=black}
\setbeamercolor{structure}{fg=black}
\setbeamercolor{block body example}{bg=black!5!white}
\setbeamercolor{block title example}{fg=white,bg=black!40!white}


\usepackage{amsmath,amssymb,amsfonts}
\usepackage{cite}
\usepackage{multirow}
\usepackage{booktabs}
\usepackage{hhline}
\usepackage{multicol}
%\usepackage{showframe}

\usepackage{tikz}
%\usetikzlibrary{patterns}
\usetikzlibrary{patterns,arrows,decorations.pathmorphing,backgrounds,shadows,positioning,fit,shapes,matrix,calc,shapes.multipart,arrows.meta}
\usepackage[simplified]{pgf-umlcd}
\usepackage{xpatch} % Needed for patching pgf-umlcd
\usepackage{xparse} % Needed for patching pgf-umlcd
\usepackage{color,soul} % for \hl
\definecolor{dark-yellow}{RGB}{219, 212, 143}
\definecolor{dark-green}{RGB}{36,84,36}
\definecolor{my-gray}{gray}{0.85}
\sethlcolor{dark-yellow}



\usepackage{wrapfig}
\usepackage{listings}
\usepackage{adjustbox}
\usepackage{graphicx}
\usepackage{caption}
\usepackage{multirow}
\usepackage{subcaption}
\usepackage{stmaryrd}
\usepackage{hyperref}
\usepackage{float}
\usepackage{textcomp}
\usepackage{tikz-qtree,tikz-qtree-compat}

%%% Style
% Font and layout
\newcommand*{\defemph}[1]{\ensuremath{\mathsf{#1}}}
\renewcommand*{\S}{Section}
\newcommand*{\sota}{state-of-the-art}
% Heavily used symbol style
\newcommand*{\agentstyle}[1]{{\ensuremath{\uppercase{\defemph{#1}}}}}
\newcommand*{\agentstyleMin}[1]{{\ensuremath{\lowercase{\defemph{#1}}}}}%}
\newcommand{\agent}[1]{%
  \ifstrequal{#1}{i}%
             {\ensuremath{\lowercase{\defemph{#1}}}}%
             {\ifstrequal{#1}{A}{\agentstyle{#1}}{%
\ifstrequal{#1}{a}{\agentstyle{#1}}{%
\ifstrequal{#1}{B}{\agentstyle{#1}}{%
\ifstrequal{#1}{b}{\agentstyle{#1}}{%
\ifstrequal{#1}{C}{\agentstyle{#1}}{%
\ifstrequal{#1}{c}{\agentstyle{#1}}{%
\ifstrequal{#1}{ag}{\agentstyleMin{#1}}{%
\ifstrequal{#1}{AG}{\agentstyleMin{#1}}{%
\ifstrequal{#1}{ag_1}{\agentstyleMin{#1}}{%
\ifstrequal{#1}{ag_2}{\agentstyleMin{#1}}{%
\ifstrequal{#1}{ag_i}{\agentstyleMin{#1}}{??
}}}}}}}}}}}}%
}
\newcommand*{\possarg}[2]{\ensuremath{\defemph{#1}(#2)}}
\newcommand*{\poss}[1]{\ensuremath{\defemph{#1}}}



%%% Operators, Actions and Syntax
% Epistemic logic operators
\newcommand*{\C}{\textbf{C}}
\newcommand*{\E}{\textbf{E}}
\newcommand*{\cAlpha}[1]{\ensuremath{\mathbf{C}_\alpha{#1}}}
\newcommand*{\eAlpha}[1]{\ensuremath{\mathbf{E}_\alpha{#1}}}
\newcommand*{\eAlphaIter}[2]{\ensuremath{\mathbf{E}^{#1}_\alpha{#2}}}
%\newcommand*{\initiallyC}[1]{\ensuremath{\texttt{initially\}(#1)}}
\newcommand*{\bB}[2]{\mathbf{B}_{\agent{#1}}{#2}}
\renewcommand*{\b}[1]{\ensuremath{\mathbf{B_{\agent{#1}}}}}
% Kripke operators
\newcommand*{\brel}[1]{\ensuremath{\calB_{\defemph{#1}}}}
\newcommand*{\rrel}[1]{\ensuremath{\calR_{\defemph{#1}}}}
% Actions and fluent
\newcommand*{\distract}[2]{%
\ifstrequal{#2}{}%
{\ensuremath{\mathtt{distract}(\agent{#1})}}%
{\ensuremath{\mathtt{distract}(\agent{#1})\langle\agent{#2}\rangle}}%
}
\newcommand*{\open}[1]{%
\ifstrequal{#1}{}%
{\ensuremath{\mathtt{open}}}%
{\ensuremath{\mathtt{open}\tuple{\agent{#1}}}}%
}
\newcommand*{\shout}[1]{%
\ifstrequal{#1}{}%
{\ensuremath{\mathtt{shout\_tails}}}%
{\ensuremath{\mathtt{shout\_tails}\tuple{\agent{#1}}}}%
}
\newcommand*{\signal}[2]{%
\ifstrequal{#2}{}%
{\ensuremath{\mathtt{signal}(\agent{#1})}}%
{\ensuremath{\mathtt{signal}(\agent{#1})\langle\agent{#2}\rangle}}%
}
\newcommand*{\peek}[1]{%
\ifstrequal{#1}{}%
{\ensuremath{\mathtt{peek}}}%
{\ensuremath{\mathtt{peek}\tuple{\agent{#1}}}}%
}
\newcommand*{\tell}[2]{%
\ifstrequal{#2}{}%
{\ensuremath{\mathtt{tell}(\agent{#1})}}%
{\ensuremath{\mathtt{tell}(\agent{#1})\langle\agent{#2}\rangle}}%
}
\newcommand*{\flip}[1]{%
\ifstrequal{#1}{}%
{\ensuremath{\mathtt{flip}}}%
{\ensuremath{\mathtt{flip}\tuple{\agent{#1}}}}%
}
\newcommand*{\haskey}[1]{\ensuremath{\mathtt{key}(\agent{#1})}}
\newcommand*{\opened}{\ensuremath{\mathtt{opened}}}
\newcommand*{\head}{\ensuremath{\mathtt{heads}}}
\newcommand*{\looking}[1]{\ensuremath{\mathtt{look}(\agent{#1})}}
\newcommand*{\res}[1]{\ensuremath{\defemph{caused}(\defemph{#1})}}
\newcommand*{\sensed}[2]{\ensuremath{\defemph{sensed}(\defemph{#1})[\defemph{#2}]}}
% Syntax
\newcommand*{\exec}[2]{\ensuremath{\mathbf{executable\ }\defemph{#1} \mathbf{\ if\ } #2 }}
\newcommand*{\causes}[2]{\ensuremath{\defemph{#1} \mathbf{\ causes\ } \defemph{#2} }}
\newcommand*{\determine}[2]{\ensuremath{\defemph{#1} \mathbf{\ determines\ } \defemph{#2} }}
\newcommand*{\announce}[2]{\ensuremath{\defemph{#1} \mathbf{\ announces\ } \defemph{#2} }}
\newcommand*{\initially}[1]{\ensuremath{\mathbf{intially}\ #1}}



%%% Special words
% Action languages
\newcommand*{\ourL}{\ensuremath{m\mathcal{A}^\rho}}
\newcommand*{\mAL}{\ensuremath{m\mathcal{A}^*}}
\newcommand*{\mAP}{\ensuremath{m\mathcal{A}+}}
% Multi agent epistemic
\newcommand*{\mAGep}{multi-agent epistemic}
\newcommand*{\MAGep}{Multi-agent epistemic}
\newcommand*{\ck}{common knowledge}
\newcommand*{\mep}{MEP}
% Non-well-founded set
\newcommand*{\Wf}{Well-founded}
\newcommand*{\wf}{well-founded}
\newcommand*{\Nwf}{Non-\wf}
\newcommand*{\nwf}{non-\wf}
%Possibilities & Co.
\newcommand*{\sPoss}{\ensuremath{\Phi_{\mathbf{Poss}}}}
\newcommand*{\Pos}{Possibility}
\newcommand*{\pos}{possibility}
\newcommand*{\PosS}{Possibilities}
\newcommand*{\posS}{possibilities}



%%%Special Symbols
% Languages
\newcommand*{\lAG}{\ensuremath{\mathcal{L}_{\sAG}}}
\newcommand*{\lag}{\lAG}
\newcommand*{\lagC}{\ensuremath{\lAG^{\C}}}
% Kripke structures
\newcommand*{\state}[2]{\ensuremath{(M_{\defemph{#1}},\defemph{#2})}}
\newcommand*{\posfunc}[3]{\ensuremath{\Phi_{D_{#1}}(#2,\defemph{#3})}}
\newcommand*{\trfunc}{\ensuremath{\Phi_D}}
\newcommand*{\posfull}{\ensuremath{F_{D}}}
\newcommand*{\pospartial}{\ensuremath{P_{D}}}
\newcommand*{\posoblivious}{\ensuremath{O_{D}}}
\newcommand*{\interp}[2]{\ensuremath{M_{#1}[\pi](\defemph #2)}}
% Special Sets
\newcommand*{\sAG}{\ensuremath{\mathcal{AG}}}
\newcommand*{\sAC}{\ensuremath{\mathcal{A}}}
\newcommand*{\sF}{\ensuremath{\mathcal{F}}}
\newcommand*{\sP}{\sF}
\newcommand*{\ai}{\calA\calI}
% Graph
\newcommand*{\graphG}{\ensuremath{\mathcal{G}}}
\newcommand*{\graphVE}[2]{\graphG=\textup{(}$#1, #2$\textup{)}}
% Generic
\newcommand*{\func}[3]{#1: #2 \mapsto #3}
\renewcommand*{\implies}{\ensuremath{\Rightarrow}}



%%% Shortcuts
\newcommand*{\bra}[1]{\ensuremath{\{#1\}}}
\newcommand*{\tuple}[1]{\ensuremath{\langle #1 \rangle}}



%%% Calligraphics macros by E. Zaffanella
\newcommand*{\calA}{\ensuremath{\mathcal{A}}}
\newcommand*{\calB}{\ensuremath{\mathcal{B}}}
\newcommand*{\calC}{\ensuremath{\mathcal{C}}}
\newcommand*{\calD}{\ensuremath{\mathcal{D}}}
\newcommand*{\calE}{\ensuremath{\mathcal{E}}}
\newcommand*{\calF}{\ensuremath{\mathcal{F}}}
\newcommand*{\calG}{\ensuremath{\mathcal{G}}}
\newcommand*{\calH}{\ensuremath{\mathcal{H}}}
\newcommand*{\calI}{\ensuremath{\mathcal{I}}}
\newcommand*{\calJ}{\ensuremath{\mathcal{J}}}
\newcommand*{\calK}{\ensuremath{\mathcal{K}}}
\newcommand*{\calL}{\ensuremath{\mathcal{L}}}
\newcommand*{\calM}{\ensuremath{\mathcal{M}}}
\newcommand*{\calN}{\ensuremath{\mathcal{N}}}
\newcommand*{\calO}{\ensuremath{\mathcal{O}}}
\newcommand*{\calP}{\ensuremath{\mathcal{P}}}
\newcommand*{\calQ}{\ensuremath{\mathcal{Q}}}
\newcommand*{\calR}{\ensuremath{\mathcal{R}}}
\newcommand*{\calS}{\ensuremath{\mathcal{S}}}
\newcommand*{\calT}{\ensuremath{\mathcal{T}}}
\newcommand*{\calU}{\ensuremath{\mathcal{U}}}
\newcommand*{\calV}{\ensuremath{\mathcal{V}}}
\newcommand*{\calW}{\ensuremath{\mathcal{W}}}
\newcommand*{\calX}{\ensuremath{\mathcal{X}}}
\newcommand*{\calY}{\ensuremath{\mathcal{Y}}}
\newcommand*{\calZ}{\ensuremath{\mathcal{Z}}}



%% Checkmark
\def\checkmark{\tikz\fill[scale=0.4](0,.35) -- (.25,0) -- (1,.7) -- (.25,.15) -- cycle;}

%%For SLIDES only
\newcommand*{\emphColorSlide}[1]{\textcolor{ForestGreen}{#1}}
\newcommand*{\emphSlide}[1]{\emphColorSlide{\emph{#1}}}
\newcommand*{\ttSlide}[1]{\textcolor{NavyBlue}{\texttt{#1}}}

\newcommand*{\colorAgentSlide}[1]{\textcolor{Black}{#1}}
\newcommand*{\agentSlide}[1]{%
\ifstrequal{#1}{Charlie}{\colorAgentSlide{\texttt{#1}}}%
{\ifstrequal{#1}{Lucy}{\colorAgentSlide{\texttt{#1}}}%
{\ifstrequal{#1}{Snoopy}{\colorAgentSlide{\texttt{#1}}}%
{\ifstrequal{#1}{ag}{\colorAgentSlide{\texttt{#1}}}%
{\ifstrequal{#1}{ag_i}{\ensuremath{\colorAgentSlide{\mathtt{#1}}}}%
{\ifstrequal{#1}{ag_1}{\ensuremath{\colorAgentSlide{\mathtt{#1}}}}%
{\ifstrequal{#1}{ag_2}{\ensuremath{\colorAgentSlide{\mathtt{#1}}}}%
{\ifstrequal{#1}{AG}{\colorAgentSlide{\texttt{\lowercase{#1}}}}%
{\ifstrequal{#1}{A}{\colorAgentSlide{\texttt{#1}}}%
{\ifstrequal{#1}{B}{\colorAgentSlide{\texttt{#1}}}%
{\ifstrequal{#1}{C}{\colorAgentSlide{\texttt{#1}}}%
{\ifstrequal{#1}{a}{\colorAgentSlide{\uppercase{\texttt{#1}}}}%
{\ifstrequal{#1}{b}{\colorAgentSlide{\uppercase{\texttt{#1}}}}%
{\ifstrequal{#1}{c}{\colorAgentSlide{\uppercase{\texttt{#1}}}}%
{\ifstrequal{#1}{agent}{\colorAgentSlide{\texttt{#1}}}%
{\ifstrequal{#1}{agents}{\colorAgentSlide{\texttt{#1}}}{??%
}}}}}}}}}}}}}}}}%
}
\newcommand*{\resSlide}[2]{\defemph{caused(#1)}[\ttSlide{#2}]}
\newcommand*{\sensedSlide}[2]{\defemph{sensed(#1)}[\ttSlide{#2}]}
\newcommand*{\brelSlide}[1]{\ensuremath{\calR_{\defemph{#1}}}}
\newcommand*{\bBSlide}[2]{\mathbf{B}_{\texttt{#1}}{#2}}
\newcommand{\showCILC}[2]{%
	\ifstrequal{#1}{true}{#2}{}}


%\full (Azione, pointed kripke structure(2 argomenti), insieme  )
%\newcommand*{\Interp}[2]{\ensuremath{M_{#1}[\pi](#2)}}
%\newcommand*{\nwfeq}[3]{\ensuremath{#1(#2)=#3}}



\makeatletter
\renewcommand\@makefnmark{\hbox{\@textsuperscript{\usebeamercolor[fg]{footnote mark}\usebeamerfont*{footnote mark}[\@thefnmark]}}}
\renewcommand\@makefntext[1]{\@textsuperscript{\usebeamercolor[fg]{footnote mark}\usebeamerfont*{footnote mark}[\@thefnmark]}\usebeamerfont*{footnote} #1}
\makeatother

%Modify the Title display on frame
\makeatletter
\patchcmd\beamer@@tmpl@frametitle{\insertframetitle}{{\footnotesize \insertsection~\emphColorSlide{\insertsubsection}~\\} \textbf{\insertframetitle}}{}{}
\makeatother
%Add space to the footline
\makeatletter
\patchcmd{\beamer@calculateheadfoot}{\advance\footheight by 4pt}{\advance\footheight by 8pt}{}{}
\makeatother

%NO HEAD LINE
\makeatletter
\newenvironment{noheadline}{
	\setbeamertemplate{headline}[default] {}
	\setfootline
}{}
\makeatother

%Slide with  Section title
%\AtBeginSection[]{
%	\begin{noheadline}
%	\begin{frame}[noframenumbering]
%			\begin{center}
%				\large Chapter \thesection \\
%				\vspace*{1cm}
%				\centering {\usebeamerfont{title} \huge\textcolor{black}{\insertsectionhead}}
%				\vspace*{1cm}
%			\end{center}
%		\end{frame}
%	\end{noheadline}
%}


\newcommand{\setfootline}{
	\setbeamercolor{footline}{fg=white,bg=black}
	\setbeamertemplate{footline}{%
		\begin{beamercolorbox}[wd=1.0\paperwidth,left,ht=2.5ex,dp=1ex]{footline}
			\usebeamerfont{section in head/foot}%
			\hspace*{3.5ex}%
			\insertshortauthor\ |\
			\insertshorttitle
			\insertshortsubtitle
		\end{beamercolorbox}
	}
	\addtobeamertemplate{footline}{%
		\setlength\unitlength{1ex}%
		\begin{picture}(0,0)
		\put(125,3){\makebox(0,0)[bl]{
			\includegraphics[scale=0.20]{img/scam.png}
	}}%
		\end{picture}%
	}{}
}






%-------------------------------------------------------
% INFORMATION IN THE TITLE PAGE
%-------------------------------------------------------

\subtitle[{\color{ForestGreen}\textbf{TAPL 2022}}]
{
	\footnotesize{TAPL 2022
}
}

\title[] % [] is optional - is placed on the bottom of the sidebar on every slide
{ % is placed on the title page
	\textsc{
	 Type Reconstruction}
}



\author[\textbf{Idriss Riouak}]
{ \emphSlide{Idriss Riouak}}

\institute[]
{\vspace{1cm}
	\textbf{Lund University}\\
	\textbf{Computer Science Department} \\
}

\date{\today}

\makeatletter
\newcommand\SoulColor{%
  \let\set@color\beamerorig@set@color
  \let\reset@color\beamerorig@reset@color}
\makeatother
\SoulColor

\begin{document}
\setfootline

%-------------------------------------------------------
% THE TITLEPAGE
%-------------------------------------------------------

{\1
\begin{frame}[plain,noframenumbering]
		\titlepage
	\end{frame}}

%-------------------------------------------------------
% BODY
%-------------------------------------------------------

\begin{frame}{Ch. 22 - Type reconstruction}
\hspace{-1cm}
\begin{minipage}[t]{0.4\textwidth}
\includegraphics[scale=0.25]{img/depgraph}
\end{minipage}
\begin{minipage}[t]{0.4\textwidth}
\vspace{-5cm}
\includegraphics[scale=0.17]{img/1}
\end{minipage}
\end{frame}


\begin{frame}{Goal of today lecture}

\emphColorSlide{Type checking}: Given $\Gamma,t$ and $T$, \alert{check} whether $\Gamma \models t:T$.


\vspace{1cm}


\emphColorSlide{Type reconstruction}: Given $\Gamma$ and $t$, \alert{find} a type $T$ s.t. $\Gamma \models t:T$.
\end{frame}

\begin{frame}{Outline}
\includegraphics[scale=0.25]{img/2}
\end{frame}

\begin{frame}{Operations on type variables}
Let us consider the following term:
$$\lambda g:Y. \lambda a:X. g (g\ a)$$
\alert{it is not typable} as it stands.
\onslide<2->{
It is typable if we \emphColorSlide{substitute} $Y$ with $\mathtt{Nat} \rightarrow \mathtt{Nat}$ and $X$ with $\mathtt{Nat}$:
$$\lambda g:\mathtt{Nat} \rightarrow \mathtt{Nat}.  \lambda a:\mathtt{Nat}. g (g\ a)$$
}
\vspace*{-0.5cm}
\onslide<3->{
\begin{block}{Definition}
A type substitution is a finite mapping from type variables to types.
\end{block}}
\onslide<4->{
\begin{example}{}
The substitution $[Y \mapsto X \rightarrow X, X \mapsto \mathtt{Bool}]$ will map $X$ to $\mathtt{Bool}$, and $Y$ to $X \rightarrow X$, not $ \mathtt{Bool} \rightarrow  \mathtt{Bool}$
\end{example}
}
\end{frame}

\begin{frame}{Operation on type variables}
Application of a substitution $\sigma$ to a type:
\begin{align*}
\sigma(X) &= \begin{cases}
T & \text{if } (X \mapsto T) \in \sigma\\
X & \text{if } X \not\in dom(\sigma)
\end{cases}\\
\sigma(Nat) &= Nat\\
\sigma(Bool) &= Bool\\
\sigma(T_1 \rightarrow T_2) &= \sigma(T_1) \rightarrow \sigma(T_2)
\end{align*}


If $\sigma$ and $\gamma$ are substitutions:
$$\sigma \circ \gamma = \begin{cases}
X \mapsto \sigma(T)& \text{for each } (X \mapsto T)\in \gamma\\
X \mapsto T & \text{for each } (X \mapsto T) \in\sigma \wedge X \not\in dom(\gamma)
\end{cases} $$

TL;DR: $(\sigma \circ \gamma )S=\sigma(\gamma S)$
\end{frame}

\begin{frame}{Outline}
\includegraphics[scale=0.25]{img/3}
\end{frame}
\begin{frame}{Two interesting questions}
\begin{itemize}
\item ``Are \alert{all} substitution instances of $t$ well typed ?''\\
  i.e., $\forall \sigma. \sigma\Gamma \models \sigma t: T$ for some $T$ ?
\begin{example}
$\lambda g: X \rightarrow X.  \lambda a: X. g(g\ a)$
\end{example}
\item ``Is \emphColorSlide{some} substitution instance of $t$ well typed'' ? \\
  i.e., $\exists \sigma: \sigma\Gamma \models \sigma t: T$ for some $T$ ?
\begin{example}
$\lambda g: Y \rightarrow X.  \lambda a: X. g(g\ a)$
\end{example}

\end{itemize}
Looking for \emphColorSlide{valid instantiations} of type variables leads to the ideas of type \textit{reconstruction} (or type inference). The programmer (as in ML or Haskell) may leave out all type annotations.
\end{frame}


\begin{frame}{Type solution}
\begin{block}{Definition}
Let $\Gamma$ be a context and $t$ a term. A \emphColorSlide{solution} for $(\Gamma,t)$ is a pair $(\alert{\sigma},\emphColorSlide{T})$ s.t. $\alert{\sigma}\Gamma\models\alert{\sigma} t:\emphColorSlide{T}$.
\end{block}

\begin{example}
Let $\Gamma=g:X, a:Y$ and $t=g\ a.$ Then all the solutions for $(\Gamma,t)$ are:
\begin{itemize}
\item $([X \mapsto Y \rightarrow \mathtt{Nat}],\mathtt{Nat})$
\item $([X \mapsto Y \rightarrow  Z, Z \mapsto \mathtt{Nat}],Z)$
\item $([X \mapsto \mathtt{Nat} \rightarrow \mathtt{Nat}, Y \mapsto \mathtt{Nat}],\mathtt{Nat})$
\item $([X \mapsto Y \rightarrow Z],Z)$
\item $([X \mapsto Y \rightarrow \mathtt{Nat} \rightarrow \mathtt{Nat}],\mathtt{Nat})$
\end{itemize}
\end{example}


\end{frame}
\begin{frame}{Outline}
\includegraphics[scale=0.25]{img/4}
\end{frame}

\begin{frame}{Constraint-based typing}
\begin{block}{Definition}
\begin{itemize}
\item A \emphColorSlide{constraint set} C is a set of equations \{ $S_i = T_i^{i \in 1...n} $ \}.
\item A substitution $\sigma$ is said to \alert{unify} an equation $S=T$ if the substitution instances $\sigma S$ and $\sigma T$ are \emphColorSlide{identical}.
\item  We say that $\sigma$ unifies $C$ if it unifies \emphColorSlide{every equation} in C.
\end{itemize}
\end{block}

\begin{block}{Definition}
Suppose that $\Gamma \models t: S \mid_{\chi}C.$ A solution for $(\Gamma, t, S, C)$ is a pair $(\sigma, T)$ s.t. $\sigma$ satisfies $C$ and $\sigma S= T$.
\end{block}
 We read $\Gamma \models t :T \mid_\chi C$ as ``Term $t$ has type $T$ under assumptions $\Gamma$ whenever constraints $C$ are satisfied''
\end{frame}

\begin{frame}{Constraint typing rules}
\includegraphics[scale=0.4]{img/rules}
\end{frame}

\begin{frame}{Declarative and Algorithmic specification}
Given a context $\Gamma$ and a term $t$, there are two ways of instantiating type variables in $\Gamma$ and $t$ to produce a valid typing:

\vspace{0.5cm}
\begin{enumerate}
\item[1)] \emphColorSlide{[Declarative]}: as the set of all solutions for $(\Gamma,t)$ in the sense of Definition 22.2.1
\item[2)] \emphColorSlide{[Algorithmic]}: via the constraint typing relation, by finding $S$ and $C$ such that $\Gamma \models t: S\mid C$ and then taking the set of solutions for $(\Gamma, t, S,C)$.
\end{enumerate}

\vspace{.5cm}
The two specification are equivalent.

\vspace{.5cm}
You can find the proof in the book :)

\end{frame}

\begin{frame}{Outline}
\includegraphics[scale=0.25]{img/5}
\end{frame}

\begin{frame}{Unification}
To compute solutions to constraint sets, we use the idea of unification\\
\vspace{0.5cm}
\includegraphics[scale=0.4]{img/unification}
\end{frame}

\begin{frame}{More about unification}
\begin{itemize}
\item We say that a substitution $\sigma$ is \emphColorSlide{less specific} than a substitution $\sigma'$, written $\sigma \sqsubseteq \sigma'$, if $\sigma' =  \gamma \circ \sigma$ for some substitution $\gamma$.
\item $\sigma$ is a \emphColorSlide{principal unifier} for a constraint set $C$ iff $$\forall \sigma' \in C: \sigma \sqsubseteq \sigma'.$$

\begin{theorem}
The algorithm `\emphColorSlide{unify}' always terminates, failing when given a non unifiable constraint set as input and otherwise returning a principal unifier.
\end{theorem}
\end{itemize}
\end{frame}

\begin{frame}{Outline}
\includegraphics[scale=0.25]{img/6}
\end{frame}
\begin{frame}[fragile]{Let-polymorphism}
The term polymorphism refers to a range of language mechanisms that allow a single part of a program to be used with different types.

\begin{example}
\begin{lstlisting}[frame=none]
let f = \x.x in
let nat = f 1 in
let bool = f true in ...
\end{lstlisting}
\end{example}

\begin{itemize}
\item With the type reconstruction algorithms discussed previously, this program is not well-typed.
\item The type reconstruction algorithm shown before can be generalized to provide a simple form of polymorphism known as let-polymorphism.
\item More about polymorphism will be discussed next week.
\end{itemize}
\end{frame}

\begin{frame}{Let-polymorphism}
Old typing rule:
\begin{center}
\includegraphics[scale=0.4]{img/old}
\end{center}

New  typing rule:
\begin{center}
\includegraphics[scale=0.4]{img/new}
\includegraphics[scale=0.3]{img/newconstraint}
\end{center}
\onslide<2->{
What about `\texttt{let x = <garbage> in 5}' ?
}
\onslide<3->{
\begin{center}
\includegraphics[scale=0.4]{img/newnew}
\end{center}
}

\end{frame}

%\begin{frame}{Algorithmic let-polymorphism}
%The type-checking of a term \code{let x = $\mathtt{t_1}$ in\ $\mathtt{t_2}$} in a context $\Gamma$ proceeds as follows:
%\begin{itemize}
%\item Calculate a type $S_1$ and a set $C_1$ of associated constraints for the right-hand side $t_1$.
%\item Obtain $t_1$'s principal type $T_1$
%\item If $X_1...X_n$ are the remaining variables in $T_1$, we write $\forall X_1...X_n.T_1 $ for the principal type scheme of $t_1$.
%\item Each time we encounter an occurrence of $x$ in $t_2$, we generate fresh $Y_1...Y_n$, yielding [$X_1\mapsto Y_1, ..., X_n \mapsto Y_n$ ]$T_1$.
%\end{itemize}
%
%\end{frame}

\begin{frame}{Essentially linear}
The algorithm is efficient ad in practice it appears ``\emphColorSlide{essentially linear}'' in the size of the input program. But the worst-case complexity is still \alert{exponential}.

\vspace{0.5cm}
\includegraphics[scale=0.4]{img/expo}

\end{frame}

\begin{frame}{Conclusion}
\begin{center}
Type reconstruction algorithm \\ =\\ \emphColorSlide{Constraint generation} \\+\\\emphColorSlide{unification}.
\end{center}


\vspace{1cm}

\begin{center}
First introduction to\emphColorSlide{ polymorphism}
\end{center}

\end{frame}

%No page Numbering
\setbeamercolor{headline}{fg=white,bg=black}
\setbeamertemplate{headline}{
	\begin{beamercolorbox}[wd=1.0\paperwidth,left,ht=38pt,dp=1ex]{headline}
	\end{beamercolorbox}
	\vspace*{2.3pt}
	\color{black!30!white}\rule{\paperwidth}{2pt}
}


\end{document}