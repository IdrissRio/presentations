\documentclass[usenames,dvipsnames]{beamer}
%-------------------------------------------------------
% THEME SETTINGS
%-------------------------------------------------------
\usetheme[progressstyle=movingCircCnt]{Feather}
\setbeamercolor{Feather}{fg=black!30,bg=black}
\setbeamercolor{structure}{fg=black}
\setbeamercolor{block body example}{bg=black!5!white}
\setbeamercolor{block title example}{fg=white,bg=black!40!white}


\usepackage{amsmath,amssymb,amsfonts}
\usepackage{cite}
\usepackage{multirow}
\usepackage{booktabs}
\usepackage{hhline}
\usepackage{multicol}
%\usepackage{showframe}

\usepackage{tikz}
%\usetikzlibrary{patterns}
\usetikzlibrary{patterns,arrows,decorations.pathmorphing,backgrounds,shadows,positioning,fit,shapes,matrix,calc,shapes.multipart,arrows.meta}
\usepackage[simplified]{pgf-umlcd}
\usepackage{xpatch} % Needed for patching pgf-umlcd
\usepackage{xparse} % Needed for patching pgf-umlcd
\usepackage{color,soul} % for \hl
\definecolor{dark-yellow}{RGB}{219, 212, 143}
\definecolor{dark-green}{RGB}{36,84,36}
\definecolor{my-gray}{gray}{0.85}
\sethlcolor{dark-yellow}



\usepackage{wrapfig}
\usepackage{listings}
\usepackage{adjustbox}
\usepackage{graphicx}
\usepackage{caption}
\usepackage{multirow}
\usepackage{subcaption}
\usepackage{stmaryrd}
\usepackage{hyperref}
\usepackage{float}
\usepackage{textcomp}
\usepackage{tikz-qtree,tikz-qtree-compat}

%%% Style
% Font and layout
\newcommand*{\defemph}[1]{\ensuremath{\mathsf{#1}}}
\renewcommand*{\S}{Section}
\newcommand*{\sota}{state-of-the-art}
% Heavily used symbol style
\newcommand*{\agentstyle}[1]{{\ensuremath{\uppercase{\defemph{#1}}}}}
\newcommand*{\agentstyleMin}[1]{{\ensuremath{\lowercase{\defemph{#1}}}}}%}
\newcommand{\agent}[1]{%
  \ifstrequal{#1}{i}%
             {\ensuremath{\lowercase{\defemph{#1}}}}%
             {\ifstrequal{#1}{A}{\agentstyle{#1}}{%
\ifstrequal{#1}{a}{\agentstyle{#1}}{%
\ifstrequal{#1}{B}{\agentstyle{#1}}{%
\ifstrequal{#1}{b}{\agentstyle{#1}}{%
\ifstrequal{#1}{C}{\agentstyle{#1}}{%
\ifstrequal{#1}{c}{\agentstyle{#1}}{%
\ifstrequal{#1}{ag}{\agentstyleMin{#1}}{%
\ifstrequal{#1}{AG}{\agentstyleMin{#1}}{%
\ifstrequal{#1}{ag_1}{\agentstyleMin{#1}}{%
\ifstrequal{#1}{ag_2}{\agentstyleMin{#1}}{%
\ifstrequal{#1}{ag_i}{\agentstyleMin{#1}}{??
}}}}}}}}}}}}%
}
\newcommand*{\possarg}[2]{\ensuremath{\defemph{#1}(#2)}}
\newcommand*{\poss}[1]{\ensuremath{\defemph{#1}}}



%%% Operators, Actions and Syntax
% Epistemic logic operators
\newcommand*{\C}{\textbf{C}}
\newcommand*{\E}{\textbf{E}}
\newcommand*{\cAlpha}[1]{\ensuremath{\mathbf{C}_\alpha{#1}}}
\newcommand*{\eAlpha}[1]{\ensuremath{\mathbf{E}_\alpha{#1}}}
\newcommand*{\eAlphaIter}[2]{\ensuremath{\mathbf{E}^{#1}_\alpha{#2}}}
%\newcommand*{\initiallyC}[1]{\ensuremath{\texttt{initially\}(#1)}}
\newcommand*{\bB}[2]{\mathbf{B}_{\agent{#1}}{#2}}
\renewcommand*{\b}[1]{\ensuremath{\mathbf{B_{\agent{#1}}}}}
% Kripke operators
\newcommand*{\brel}[1]{\ensuremath{\calB_{\defemph{#1}}}}
\newcommand*{\rrel}[1]{\ensuremath{\calR_{\defemph{#1}}}}
% Actions and fluent
\newcommand*{\distract}[2]{%
\ifstrequal{#2}{}%
{\ensuremath{\mathtt{distract}(\agent{#1})}}%
{\ensuremath{\mathtt{distract}(\agent{#1})\langle\agent{#2}\rangle}}%
}
\newcommand*{\open}[1]{%
\ifstrequal{#1}{}%
{\ensuremath{\mathtt{open}}}%
{\ensuremath{\mathtt{open}\tuple{\agent{#1}}}}%
}
\newcommand*{\shout}[1]{%
\ifstrequal{#1}{}%
{\ensuremath{\mathtt{shout\_tails}}}%
{\ensuremath{\mathtt{shout\_tails}\tuple{\agent{#1}}}}%
}
\newcommand*{\signal}[2]{%
\ifstrequal{#2}{}%
{\ensuremath{\mathtt{signal}(\agent{#1})}}%
{\ensuremath{\mathtt{signal}(\agent{#1})\langle\agent{#2}\rangle}}%
}
\newcommand*{\peek}[1]{%
\ifstrequal{#1}{}%
{\ensuremath{\mathtt{peek}}}%
{\ensuremath{\mathtt{peek}\tuple{\agent{#1}}}}%
}
\newcommand*{\tell}[2]{%
\ifstrequal{#2}{}%
{\ensuremath{\mathtt{tell}(\agent{#1})}}%
{\ensuremath{\mathtt{tell}(\agent{#1})\langle\agent{#2}\rangle}}%
}
\newcommand*{\flip}[1]{%
\ifstrequal{#1}{}%
{\ensuremath{\mathtt{flip}}}%
{\ensuremath{\mathtt{flip}\tuple{\agent{#1}}}}%
}
\newcommand*{\haskey}[1]{\ensuremath{\mathtt{key}(\agent{#1})}}
\newcommand*{\opened}{\ensuremath{\mathtt{opened}}}
\newcommand*{\head}{\ensuremath{\mathtt{heads}}}
\newcommand*{\looking}[1]{\ensuremath{\mathtt{look}(\agent{#1})}}
\newcommand*{\res}[1]{\ensuremath{\defemph{caused}(\defemph{#1})}}
\newcommand*{\sensed}[2]{\ensuremath{\defemph{sensed}(\defemph{#1})[\defemph{#2}]}}
% Syntax
\newcommand*{\exec}[2]{\ensuremath{\mathbf{executable\ }\defemph{#1} \mathbf{\ if\ } #2 }}
\newcommand*{\causes}[2]{\ensuremath{\defemph{#1} \mathbf{\ causes\ } \defemph{#2} }}
\newcommand*{\determine}[2]{\ensuremath{\defemph{#1} \mathbf{\ determines\ } \defemph{#2} }}
\newcommand*{\announce}[2]{\ensuremath{\defemph{#1} \mathbf{\ announces\ } \defemph{#2} }}
\newcommand*{\initially}[1]{\ensuremath{\mathbf{intially}\ #1}}



%%% Special words
% Action languages
\newcommand*{\ourL}{\ensuremath{m\mathcal{A}^\rho}}
\newcommand*{\mAL}{\ensuremath{m\mathcal{A}^*}}
\newcommand*{\mAP}{\ensuremath{m\mathcal{A}+}}
% Multi agent epistemic
\newcommand*{\mAGep}{multi-agent epistemic}
\newcommand*{\MAGep}{Multi-agent epistemic}
\newcommand*{\ck}{common knowledge}
\newcommand*{\mep}{MEP}
% Non-well-founded set
\newcommand*{\Wf}{Well-founded}
\newcommand*{\wf}{well-founded}
\newcommand*{\Nwf}{Non-\wf}
\newcommand*{\nwf}{non-\wf}
%Possibilities & Co.
\newcommand*{\sPoss}{\ensuremath{\Phi_{\mathbf{Poss}}}}
\newcommand*{\Pos}{Possibility}
\newcommand*{\pos}{possibility}
\newcommand*{\PosS}{Possibilities}
\newcommand*{\posS}{possibilities}



%%%Special Symbols
% Languages
\newcommand*{\lAG}{\ensuremath{\mathcal{L}_{\sAG}}}
\newcommand*{\lag}{\lAG}
\newcommand*{\lagC}{\ensuremath{\lAG^{\C}}}
% Kripke structures
\newcommand*{\state}[2]{\ensuremath{(M_{\defemph{#1}},\defemph{#2})}}
\newcommand*{\posfunc}[3]{\ensuremath{\Phi_{D_{#1}}(#2,\defemph{#3})}}
\newcommand*{\trfunc}{\ensuremath{\Phi_D}}
\newcommand*{\posfull}{\ensuremath{F_{D}}}
\newcommand*{\pospartial}{\ensuremath{P_{D}}}
\newcommand*{\posoblivious}{\ensuremath{O_{D}}}
\newcommand*{\interp}[2]{\ensuremath{M_{#1}[\pi](\defemph #2)}}
% Special Sets
\newcommand*{\sAG}{\ensuremath{\mathcal{AG}}}
\newcommand*{\sAC}{\ensuremath{\mathcal{A}}}
\newcommand*{\sF}{\ensuremath{\mathcal{F}}}
\newcommand*{\sP}{\sF}
\newcommand*{\ai}{\calA\calI}
% Graph
\newcommand*{\graphG}{\ensuremath{\mathcal{G}}}
\newcommand*{\graphVE}[2]{\graphG=\textup{(}$#1, #2$\textup{)}}
% Generic
\newcommand*{\func}[3]{#1: #2 \mapsto #3}
\renewcommand*{\implies}{\ensuremath{\Rightarrow}}



%%% Shortcuts
\newcommand*{\bra}[1]{\ensuremath{\{#1\}}}
\newcommand*{\tuple}[1]{\ensuremath{\langle #1 \rangle}}



%%% Calligraphics macros by E. Zaffanella
\newcommand*{\calA}{\ensuremath{\mathcal{A}}}
\newcommand*{\calB}{\ensuremath{\mathcal{B}}}
\newcommand*{\calC}{\ensuremath{\mathcal{C}}}
\newcommand*{\calD}{\ensuremath{\mathcal{D}}}
\newcommand*{\calE}{\ensuremath{\mathcal{E}}}
\newcommand*{\calF}{\ensuremath{\mathcal{F}}}
\newcommand*{\calG}{\ensuremath{\mathcal{G}}}
\newcommand*{\calH}{\ensuremath{\mathcal{H}}}
\newcommand*{\calI}{\ensuremath{\mathcal{I}}}
\newcommand*{\calJ}{\ensuremath{\mathcal{J}}}
\newcommand*{\calK}{\ensuremath{\mathcal{K}}}
\newcommand*{\calL}{\ensuremath{\mathcal{L}}}
\newcommand*{\calM}{\ensuremath{\mathcal{M}}}
\newcommand*{\calN}{\ensuremath{\mathcal{N}}}
\newcommand*{\calO}{\ensuremath{\mathcal{O}}}
\newcommand*{\calP}{\ensuremath{\mathcal{P}}}
\newcommand*{\calQ}{\ensuremath{\mathcal{Q}}}
\newcommand*{\calR}{\ensuremath{\mathcal{R}}}
\newcommand*{\calS}{\ensuremath{\mathcal{S}}}
\newcommand*{\calT}{\ensuremath{\mathcal{T}}}
\newcommand*{\calU}{\ensuremath{\mathcal{U}}}
\newcommand*{\calV}{\ensuremath{\mathcal{V}}}
\newcommand*{\calW}{\ensuremath{\mathcal{W}}}
\newcommand*{\calX}{\ensuremath{\mathcal{X}}}
\newcommand*{\calY}{\ensuremath{\mathcal{Y}}}
\newcommand*{\calZ}{\ensuremath{\mathcal{Z}}}



%% Checkmark
\def\checkmark{\tikz\fill[scale=0.4](0,.35) -- (.25,0) -- (1,.7) -- (.25,.15) -- cycle;}

%%For SLIDES only
\newcommand*{\emphColorSlide}[1]{\textcolor{ForestGreen}{#1}}
\newcommand*{\emphSlide}[1]{\emphColorSlide{\emph{#1}}}
\newcommand*{\ttSlide}[1]{\textcolor{NavyBlue}{\texttt{#1}}}

\newcommand*{\colorAgentSlide}[1]{\textcolor{Black}{#1}}
\newcommand*{\agentSlide}[1]{%
\ifstrequal{#1}{Charlie}{\colorAgentSlide{\texttt{#1}}}%
{\ifstrequal{#1}{Lucy}{\colorAgentSlide{\texttt{#1}}}%
{\ifstrequal{#1}{Snoopy}{\colorAgentSlide{\texttt{#1}}}%
{\ifstrequal{#1}{ag}{\colorAgentSlide{\texttt{#1}}}%
{\ifstrequal{#1}{ag_i}{\ensuremath{\colorAgentSlide{\mathtt{#1}}}}%
{\ifstrequal{#1}{ag_1}{\ensuremath{\colorAgentSlide{\mathtt{#1}}}}%
{\ifstrequal{#1}{ag_2}{\ensuremath{\colorAgentSlide{\mathtt{#1}}}}%
{\ifstrequal{#1}{AG}{\colorAgentSlide{\texttt{\lowercase{#1}}}}%
{\ifstrequal{#1}{A}{\colorAgentSlide{\texttt{#1}}}%
{\ifstrequal{#1}{B}{\colorAgentSlide{\texttt{#1}}}%
{\ifstrequal{#1}{C}{\colorAgentSlide{\texttt{#1}}}%
{\ifstrequal{#1}{a}{\colorAgentSlide{\uppercase{\texttt{#1}}}}%
{\ifstrequal{#1}{b}{\colorAgentSlide{\uppercase{\texttt{#1}}}}%
{\ifstrequal{#1}{c}{\colorAgentSlide{\uppercase{\texttt{#1}}}}%
{\ifstrequal{#1}{agent}{\colorAgentSlide{\texttt{#1}}}%
{\ifstrequal{#1}{agents}{\colorAgentSlide{\texttt{#1}}}{??%
}}}}}}}}}}}}}}}}%
}
\newcommand*{\resSlide}[2]{\defemph{caused(#1)}[\ttSlide{#2}]}
\newcommand*{\sensedSlide}[2]{\defemph{sensed(#1)}[\ttSlide{#2}]}
\newcommand*{\brelSlide}[1]{\ensuremath{\calR_{\defemph{#1}}}}
\newcommand*{\bBSlide}[2]{\mathbf{B}_{\texttt{#1}}{#2}}
\newcommand{\showCILC}[2]{%
	\ifstrequal{#1}{true}{#2}{}}


%\full (Azione, pointed kripke structure(2 argomenti), insieme  )
%\newcommand*{\Interp}[2]{\ensuremath{M_{#1}[\pi](#2)}}
%\newcommand*{\nwfeq}[3]{\ensuremath{#1(#2)=#3}}



\makeatletter
\renewcommand\@makefnmark{\hbox{\@textsuperscript{\usebeamercolor[fg]{footnote mark}\usebeamerfont*{footnote mark}[\@thefnmark]}}}
\renewcommand\@makefntext[1]{\@textsuperscript{\usebeamercolor[fg]{footnote mark}\usebeamerfont*{footnote mark}[\@thefnmark]}\usebeamerfont*{footnote} #1}
\makeatother

%Modify the Title display on frame
\makeatletter
\patchcmd\beamer@@tmpl@frametitle{\insertframetitle}{{\footnotesize \insertsection~\emphColorSlide{\insertsubsection}~\\} \textbf{\insertframetitle}}{}{}
\makeatother
%Add space to the footline
\makeatletter
\patchcmd{\beamer@calculateheadfoot}{\advance\footheight by 4pt}{\advance\footheight by 8pt}{}{}
\makeatother

%NO HEAD LINE
\makeatletter
\newenvironment{noheadline}{
	\setbeamertemplate{headline}[default] {}
	\setfootline
}{}
\makeatother

%Slide with  Section title
%\AtBeginSection[]{
%	\begin{noheadline}
%	\begin{frame}[noframenumbering]
%			\begin{center}
%				\large Chapter \thesection \\
%				\vspace*{1cm}
%				\centering {\usebeamerfont{title} \huge\textcolor{black}{\insertsectionhead}}
%				\vspace*{1cm}
%			\end{center}
%		\end{frame}
%	\end{noheadline}
%}


\newcommand{\setfootline}{
	\setbeamercolor{footline}{fg=white,bg=black}
	\setbeamertemplate{footline}{%
		\begin{beamercolorbox}[wd=1.0\paperwidth,left,ht=2.5ex,dp=1ex]{footline}
			\usebeamerfont{section in head/foot}%
			\hspace*{3.5ex}%
			\insertshortauthor\ |\
			\insertshorttitle
			\insertshortsubtitle
		\end{beamercolorbox}
	}
	\addtobeamertemplate{footline}{%
		\setlength\unitlength{1ex}%
		\begin{picture}(0,0)
		\put(125,3){\makebox(0,0)[bl]{
			\includegraphics[scale=0.20]{img/scam.png}
	}}%
		\end{picture}%
	}{}
}






%-------------------------------------------------------
% INFORMATION IN THE TITLE PAGE
%-------------------------------------------------------

\subtitle[{\color{ForestGreen}\textbf{TAPL}}]
{
	\footnotesize{TAPL
}
}

\title[] % [] is optional - is placed on the bottom of the sidebar on every slide
{ % is placed on the title page
	\textsc{ On Variance-Based Subtyping for Parametric Types
	 }
}



\author[\textbf{Idriss Riouak}]
{ \emphSlide{Idriss Riouak}}

\institute[]
{\vspace{1cm}
	\textbf{Lund University}\\
	\textbf{Computer Science Department} \\
}

\date{\today}

\makeatletter
\newcommand\SoulColor{%
  \let\set@color\beamerorig@set@color
  \let\reset@color\beamerorig@reset@color}
\makeatother
\SoulColor

\begin{document}
\setfootline

%-------------------------------------------------------
% THE TITLEPAGE
%-------------------------------------------------------

{\1
\begin{frame}[plain,noframenumbering]
		\titlepage
	\end{frame}}

%-------------------------------------------------------
% BODY
%-------------------------------------------------------
\section{Introduction}
	\begin{frame}[fragile]{Overview}
		\begin{itemize}
		\item ECOOP 2002 - Atsushi \textbf{Igarashi} and Mirko \textbf{Viroli}
		\item Igarashi is one of the authors of \emphSlide{Featherweight Java}
		\item \textbf{Motivations}
		\begin{itemize}[<+->]
			\item Java's designers initially decided to avoid generic features,
			\item Only inclusive polymorphism supported by inheritance,
			\item Java5: \emphSlide{$\uparrow$} programmers' productivity; \emphSlide{$\uparrow$} readability; \emphSlide{$\uparrow$}maintainability and \emphSlide{$\uparrow$}safety.
			\item Not only \textit{inheritance} subtyping but also \emphSlide{pointwise} subtyping
			\texttt{Stack<X>} \subt\ \texttt{Vector<X>} $\implies$ \texttt{Stack<String>} \subt\ \texttt{Vector<String>}
		\end{itemize}

		\end{itemize}
	\end{frame}

	\begin{frame}[fragile]{Variance}
		\begin{itemize}[<+->]
		\item Necessity to introduce another subtyping scheme: \emphSlide{variance}.
		\item Defines the subtype relation between \textbf{different} instantiations of the \textbf{same} generic class.
		\item A generic class \code{C<X>} is said to be:
			\begin{itemize}
			\item \emphSlide{covariant} w.r.t. \code{X} if $ \forall S,T:  S\subt T \implies \code{C<S>} \subt \code{C<T>}$
			\item \emphSlide{contravariant} w.r.t. \code{X} if $ \forall S,T:  S\subt T \implies \code{C<T>} \subt \code{C<S>}$
			\item \emphSlide{invariant} w.r.t. \code{X} if $ \forall S,T:   \code{C<S>} \subt \code{C<T>} \implies  S =  T$
			\end{itemize}
		\item For the type system to be \textbf{sound}, \textit{covariance} and \textit{contravariance} are permitted under some constraint on the occurrences of \code{X} within \code{C<X>}' signature:
			\begin{itemize}
				\item \emphColorSlide{covariant} if it is read-only
				\item \emphColorSlide{contravariant} if it is write-only
			\end{itemize}
		\end{itemize}
	\end{frame}



\begin{frame}[fragile]{Variance example}
\begin{lstlisting}[language=JastAdd]
class Pair<X extends Object, Y extends Object> extends Object{
  private X fst;
  private Y snd;
  Pair(X fst,Y snd){ this.fst=fst; this.snd=snd; }
  void setFst(X fst){ this.fst=fst; }
  Y getSnd(){ return snd; }
}
\end{lstlisting}
This class can be safely considered
\begin{itemize}
\item \textit{covariant} in type variable \code{Y}, and \hfill \textit{read-only}
\item  \textit{contravariant} in type variable \code{X} \hfill \textit{write-only}
\end{itemize}
\end{frame}

\begin{frame}[fragile]{Variance example cont'd}

Any type \code{Pair<R,S>} can be safely considered a subtype of \code{Pair<String,Number>} when \code{R} \supt\ \code{String} and \code{S} \subt\ \code{Number}.

\vspace{1cm}

\begin{lstlisting}[language=JastAdd]
Number getAndSet(Pair<String,Number> c, String s){
 c.setFst(s);
 return c.getSnd();
}

Number n=getAndSet( new Pair<Object,Integer>(null, new Integer(1)),"1");
\end{lstlisting}


\end{frame}

\begin{frame}[fragile]{Problem}
The type variables typically occur in such positions that forbid both covariance and contravariance
\begin{lstlisting}[language=JastAdd]
class Vector<X>{
  private X[] ar;
  Vector(int size){ ar=new X[size];}
  int size(){ return ar.length; }
  X getElementAt(int i){ return ar[i];}
  void setElementAt(X t,int i){ ar[i]=t;}
}
\end{lstlisting}
\end{frame}

	\begin{frame}[fragile]{Variance cont'd}
The main contribution of the paper:
		\begin{itemize}
			\item Specify variance of each type parameter when the type is \emphColorSlide{used}
			\item Not when the type is \alert{declared}
			\item This should:
			\begin{itemize}
				\item release the class designer from the burden of taking variance into account
				\item improve reusability.
			\end{itemize}
		\end{itemize}
	\end{frame}
\section{Informally}

\begin{frame}[fragile]{Variant Parametric Types}
Let the programmer specify within parametric types whether the type argument should be \textit{contravariant}, \textit{covariant} or \textit{invariant}
\begin{itemize}
	\item Each type parameter may be associated with a \emphColorSlide{variance annotation}
	\begin{itemize}
		\item `+': \textit{covariance} \hfill \code{Vector<+String>}
		\item `-': \textit{contravariant} \hfill \code{Vector<-String>}
		\item `*': \textit{bivariance} \hfill \code{Vector<*String>} \textit{or} \code{Vector<*>}
	\end{itemize}
\item If the outermost parametric type is without annotation then is called \emphColorSlide{invariant}.
	\begin{itemize}
		\item \code{Vector<String>}
		\item \code{Pair<Vector<+String>,Integer>}
	\end{itemize}
\end{itemize}
\end{frame}

\begin{frame}[fragile]{Simple interpretation}
An interpretation of variant parametric types is given as a set of variant types.
\begin{itemize}
	\item \code{C<+T>} = $ \{ \code{C<S>}  \mid S \subt T \}$
	\item \code{C<-T>} = $ \{ \code{C<S>}  \mid S \supt T \}$
	\item \code{C<*T>} = $ \{ \code{C<S>} \mid \forall\ S \}$ \hfill \code{C<*T>} = \code{C<*>}
    \item An invariant type correspond to a singleton
	\begin{itemize}
	\item  \code{Vector<Integer>} = $\{ \code{Vector<Integer>} \}$
	\end{itemize}
\end{itemize}
\end{frame}

\begin{frame}[fragile]{Subtyping}
\begin{center}
 \code{Integer} \subt \code{Number} \subt \code{Object}

\vspace*{0.5cm}
\includegraphics[scale=0.28]{img/subtype}
\end{center}

\end{frame}

\section{Type System}
\begin{frame}{Syntax}
\includegraphics[scale=0.28]{img/syntax}
\end{frame}
\begin{frame}{Auxiliary functions}
\centering
\includegraphics[scale=0.20]{img/type1}
\end{frame}
\begin{frame}{Judgments}
\centering
\includegraphics[scale=0.25]{img/type2}
\end{frame}
\begin{frame}{Typing}
\centering
\includegraphics[scale=0.25]{img/type3}
\end{frame}

\begin{frame}{Final check}
\centering
\includegraphics[scale=0.25]{img/type4}
\end{frame}











\end{document}