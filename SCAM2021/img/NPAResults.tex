\newcommand{\venncellB}[3][100]{%
  \ifnum#1=100\relax%
  \node[venncellstyle] at ($(#2) + (0.5, -0.5)$) {\textcolor{green!50!black}{#3}};%
  \else%
  \node[venncellstyle] at ($(#2) + (0.5, -0.5)$) {%
    \small #3 \\[-0.6em] \footnotesize %
    \ifnum#1=0\relax%
      \textcolor{red!50!black}{#1\%}%\\%
    \else%
      \textcolor{white!40!black}{#1\%}%\\%
    \fi%
  };%
  \fi%
}
\newcommand{\Xclear}[3]{\fill[vennfill, fill=white, opacity=100] #1}
\newcommand{\Xfill}[3]{\fill[vennfill, fill=#2] #1}
\newcommand{\Xdraw}[3]{\draw[venndraw] #1}
\newcommand{\Xdash}[3]{\draw[venndraw, #3, color=#2!80!black] #1}
\newcommand{\Xfilldraw}[3]{\Xfill{#1}{#2}{#3};\Xdraw{#1}{#2}{#3};}


\begin{tikzpicture}[xscale=0.66, yscale=0.58,
    venncellstyle/.style={fill=white, inner sep=1.2pt, rounded corners=1, minimum width=0.5cm, minimum height=0.4cm},
    venncellstylebg/.style={fill=white, opacity=0.5, inner sep=1.2pt, rounded corners=1, minimum width=0.6cm, minimum height=0.45cm},
      benchlabel/.style={fill=white, draw, inner sep=1pt},
      vennfill/.style={rounded corners=4, opacity=0.5},
      venndraw/.style={rounded corners=4, black, thick, opacity=0.8},
      legend/.style={draw, rounded corners=0, minimum height=0.35cm, minimum width=1.8cm, inner sep=0},
      every text node part/.style={align=center}
  ]

  % Draw the background for the legend connectors
  \begin{scope}[xshift=0cm, yshift=0.3cm]
    \draw[line width=0.08cm, -, black]
    	(0.8, 0.1) -- ++(0, -0.6);
    \draw[line width=0.08cm, -, black]
        (12.4, 0.1) -- ++(0, -0.6);
  \end{scope}

  % For each benchmark
  \foreach \benchname/\x/\y/\results in { % individual venn intersection numbers are {x/y/count/percentage}
		ANTLR/0/0/{    0/0/11/57, 1/0/4/100},
		PMD/3.4/0/{    0/0/6/100, 1/0/7/100,  2/0/3/100},
		JFC/6.8/0/{    0/0/17/82,1/0/188/92},
		FOP/10.2/0/{   0/0/29/55, 1/0/24/91}}
  {
    % shift figure to the right coordinates
    \begin{scope}[xshift=\x cm, yshift=\y cm]

    \fill[black, opacity=0.1] (-0.1,0.1) rectangle (3.1,-1.1);

    % First draw transparent fill, then draw hard boundaries w/o transparency
    \foreach \styledraw in {\Xclear, \Xfilldraw}{ % , \Xdash
      % % SonarQube
      \styledraw{(1, 0.1) rectangle (3, -1)}{SQ}{dash pattern=on 0.5pt off 0.5pt};

      % IntraJ-N
      \styledraw{(0.0, 0) rectangle (2, -1.1)}{IJH}{dash pattern=on 0.4pt off 3pt};

    }

    % Draw the individual markers
    \foreach \vennx/\venny/\count/\percent in \results {
      \venncellB[\percent]{\vennx, \venny}{\count};
    }

    % Draw benchmark name
    \node at (1.5, -1.4)
          [minimum height=0.4cm, inner sep=1.5pt] {\textbf{\textsf{\benchname}}};
    {}

    \end{scope}
  }

  \fill[opacity=0.1] (-0.1,0.3) rectangle (13.3, 1.15);

  % Draw the legend
  \begin{scope}[xshift=0cm, yshift=0.3cm]

    \node at (1.45, 0.4)
          (IJH)
          [legend, fill=white!50!IJH] {\textcolor{black!80!IJH}{{\small \tool{IntraJ}}}};

    \node at (11.75, 0.4)
          (SQ)
          [legend, fill=white!50!SQ] {\textcolor{black!80!SQ}{{\small \tool{SonarQube}}}};

    {}
    % Manual coordinates to syncronise with the background color bit from before all else is drawn
    \draw[ultra thick, -, white!50!IJH]
    	(0.8, 0.15) -- ++(0, -0.5);
    \draw[ultra thick, -, white!50!SQ]
        (12.4, 0.15) -- ++(0, -0.5);

    %% % Draw benchmark labels
    %% %[3.39, 5.53, 7.67, 9.81]
    %% \node at (3.39, 0.4)
    %%       (ANTLR)
    %%       [benchlabel] {\textbf{\textsf{ANTLR}}};
    %% \node at (5.53, 0.4)
    %%       (PMD)
    %%       [benchlabel] {\textbf{\textsf{PMD}}};
    %% \node at (7.67, 0.4)
    %%       (JFC)
    %%       [benchlabel] {\textbf{\textsf{JFC}}};
    %% \node at (9.81, 0.4)
    %%       (FOP)
    %%       [benchlabel] {\textbf{\textsf{FOP}}};
  \end{scope}

  % top label
  %% \node[draw, minimum width=13.4cm, inner sep=0]
  \node[fill=black!70!white,  minimum width=8.85cm, inner sep=1pt]
        at (6.6, 1.5)
        {\textcolor{white}{Null Pointer Analysis}};

  %% bounding box
  \draw (current bounding box.north east) -- (current bounding box.north west) -- (current bounding box.south west) -- (current bounding box.south east) -- cycle;

\end{tikzpicture}
